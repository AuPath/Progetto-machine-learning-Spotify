\chapter{Introduzione}

In questo capitolo viene introdotto il problema, il dominio di
riferimento e l'approccio adottato per la risoluzione.

\section{Descrizione del problema}
In questo elaborato consideriamo i singoli musicali. Al giorno d'oggi
le canzoni vengono spesso ascoltate dagli utenti tramite piattaforme di
streaming musicale.

L'obbiettivo del lavoro è quello di analizzare le features dei singoli
musicali così da prevedere se una canzone diventerà o meno di
successo. Nella sezione successiva verrà meglio specificato cosa si
intende per \textbf{singolo musicale di successo}
(\autoref{sec:successo}).

\subsection{Spotify}
Spotify è un servizio di riproduzione digitale in streaming di musica,
podcast e video, con accesso immediato a milioni di brani e altri
contenuti di artisti provenienti da tutto il mondo. Questa piattaforma
viene utilizzata da milioni di utenti per ascoltare canzoni e nello
specifico singoli musicali.

Da questo servizio è possibile ottenere migliaia di brani musicali,
infatti Spotify mette a disposizione una API da cui è possibile
scaricare informazioni su brani con associate alcune
caratteristiche. Pertanto oltre alle classiche informazioni di un
brano come "titolo" o "artisti" si avrà a disposizione una serie di
caratteristiche come ad esempio quanto una canzone è "energica" o
"ballabile".


\subsection{Premi per i singoli musicali}
\label{sec:successo}

Come vedremo in \autoref{sec:descrizione_spotify} il dataset ottenuto
da Spotify mette a disposizione un campo "popularity" che indica
quanto la canzone può essere considerata popolare. Tuttavia questa
caratteristica non ci sembra adeguata per identificare una canzone
come di successo oppure no.

Per questo motivo consideriamo una canzone di successo in base alle
certificazioni ottenute, rispettivamente "disco d'oro" o "disco di
platino". Queste premi sono dei riconoscimenti vinti da un brano
musicale e storicamente fanno riferimento al numero di copie vendute
da un singolo. Tuttavia con la costante crescita dell'utilizzo di
servizi per lo streming di brani musciali, da qualche anno questi
riconoscimenti vengono assegnati anche considerando il numero di
streaming sulle diverse piattaforme, tra cui Spotify.

Riteniamo che questo riconosciemnto sia una metrica oggettiva per
considerare un singolo come di successo.

\section{Approccio al problema}
Il problema viene approcciato come un \textbf{task di classificazione
  binaria}. Dato un singolo musicale vogliamo prevedere se questo sarà
di successo o no. Pertanto sviluppiamo modelli supervisionati di
machine learning partendo da un dataset etichettato, in questo modo è
possibile classificare i brani musicali.


\section{Struttura del codice}
Di seguito viene brevemente spiegata la struttura del codice, inoltre
viene sotto indicata la working directory e l'entry point del
programma.

\dirtree{%
	.1 root.
	.2 data.
	.3 raw \\ {\textit{Dataset da integrare}}.
	.3 'songs.csv' \\{\textit{Dataset integrato da dare in input agli algoritmi}}.
	.2 doc \\{\textit{Relazione progetto}}.
	.2 images\\ {\textit{Immagini prodotte in output dagli script}}.
	.2 scraper.
	.3 'combine\_data.ipynb' \\{\textit{Notebook per combinare dataframe dei vari dischi d'oro delle nazioni}}.
	.3 'wikipedia\_songs\_scraper.py' \\{\textit{Scraper per scaricare da wikipedia le informazioni riguardo le \\ certificazione dei singoli per le diverse nazioni}}.
	.3 'import\_db.sh' \\{\textit{Import del dataset in mongoDB.}}.
	.3 'record\_linkage.js' \\{\textit{Query in mongoDB per la fase di record linkage.}}.
}

In particlare gli \textbf{script in R }si trovano in:

\dirtree{%
	.1 root.
	.2 \textbf{src}.
	.3 'main.R' \\{\textit{Entrypoint script in R}}.
	.3 functions.
	.4 'preprocessing\_functions.R' \\{\textit{Funzioni per il preprocessing e visualizzazione del dataset}}.
	.4 'training\_functions.R' \\{\textit{Training modelli e plot delle performance}}.
	.4 'libraries.R' \\{\textit{Package necessari}}.
}

\textbf{N.B.}: Per come è stato progettato il codice, la working directory è la \verb|root/| e non la cartella \verb|src/|.
L'\textbf{entry point del programma} è lo script \verb|src/main.R|
